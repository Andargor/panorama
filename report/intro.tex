% File: intro.tex
% Date: Wed Jun 17 21:45:19 2015 +0800
% Author: Yuxin Wu <ppwwyyxxc@gmail.com>
\section{Introduction}
\subsection{Compilation}
Dependencies:

\begin{enumerate}
    \item gcc $ \ge$ 4.7
    \item GNU make
    \item Eigen\footnote{\url{http://eigen.tuxfamily.org}}
    \item CImg\footnote{\url{http://cimg.eu}} (already included in the repository)
    \end{enumerate}
    CImg is only used to read and write images, and it optionally depends
    on libpng, libjpeg.

Compilation:
\begin{lstlisting}
$ cd src && make
\end{lstlisting}

\subsection{Run}
Various parameters are saved in \verb|src/config.cfg|.
Here I'm introducing some of them.

Two modes of stitching are available (set/unset the option \verb|CYLINDER|)
\begin{description}
  \item[cylinder mode] When the following conditions meet, this mode usually yields better results:
\begin{itemize}
  \item Images are taken with almost-pure rotation. (as common panoramas)
	\item Images are given in the correct order. (I might fix this in the future)
  \item Images are taken with the same camera, and a good \verb|FOCAL_LENGTH| is set.
\end{itemize}

\item [general mode] This mode has no assumptions on input images. So it'll be slower.
\end{description}

%TODO See \secref{} for details on this two modes.

Some other options users may care:
\begin{itemize}
    \item \verb|FOCAL_LENGTH|: focal length of your camera in
      35mm equivalent\footnote{\url{https://en.wikipedia.org/wiki/35_mm_equivalent_focal_length}}. Only used in cylinder mode.
    \item \verb|STRAIGHTEN|: Only used in general mode. Whether to try straighten the result. Good to set when dealing with rotational panorama.
    \item \verb|CROP|: whether to crop the final image to avoid black border.
\end{itemize}
% TODO see xxx for details

Other parameters are quality-related.
The default values are generally good for images with more than 0.7 megapixels.
If your images are too small and cannot give satisfactory results,
it might be better to resize your images rather than tune the parameters.


To run the stitcher, use
\begin{lstlisting}
./image-stitching <file1> <file2> <file3> ...
\end{lstlisting}

Output file is \verb|out.png|.


In cylinder mode, the input file names need to have the correct order.

